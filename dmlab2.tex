
\documentclass{article}
\usepackage[utf8]{inputenc}
\usepackage{hyperref} % для гиперссылок
\usepackage[russian]{babel} % для русского языка
\usepackage{mathtools} % для математических выражений

\title{Дискретная математика \\ Лабораторная работа 2}

\begin{document}

\maketitle

\section{Упражнение}

Решить уравнения путем рассуждения:

\textbf{Выполнение}:
\begin{enumerate}
\item[1)] \overline{(A \Longleftrightarrow B)} \wedge \overline{(A \Longleftrightarrow C)} \rightarrow \overline{(A\Longleftrightarrow B \wedge D)} = 0
\end{enumerate}\\
\\
Чтобы это выполнялось 1-ая часть $(\overline{(A \Longleftrightarrow B)} \wedge \overline{(A \Longleftrightarrow C)})$ должна равняться 1, а 2-ая $(\overline{(A\Longleftrightarrow B \wedge D)})$ равняться 0.
Тогда А будет равняться 0, а $B = C = 1$. Тогда D будет 0 т.к. $B = 1$. \\
\\
Доказано!
\begin{enumerate}
\item[2)] (A \rightarrow C) \wedge \overline{((B \rightarrow C) \rightarrow ((A \vee B) \rightarrow C))} = 1
\end{enumerate}
\\
\\
Чтобы это выполнялось должно быть 1-ая формула $(A \rightarrow C) = 1$, 2-ая $(\overline{((B \rightarrow C) \rightarrow ((A \vee B) \rightarrow C))}) = 0$. Разложим на мелкие кусочки: $1 - (A \rightarrow C), 2 - (B \rightarrow C), 3 - ((A \vee B) \rightarrow C)$. При этом каждая формула должна соответствовать значению:\\
$(A \rightarrow C) = 1$ \\
$(B \rightarrow C) = 1$ \\
$((A \vee B) \rightarrow C) = 0$ \\
При этом находим, что $C = 0$, $A = 0$, $B = 0$. Теперь подставим все под нашу формулу:
$(0 \rightarrow 0) \wedge \overline{(0 \rightarrow 0) \rightarrow ((0 \vee 0) \rightarrow 0))} = 1$.\\
Выходит:\\
$0 = 1$ \\
\\
НЕ доказано!

\begin{enumerate}
\item[3)] \overline{(A \rightarrow \overline{B})} \rightarrow \overline{((A \vee (B \Longleftrightarrow A)) \rightarrow C)} = 1
\end{enumerate} \\
\\
Чтобы выполнялось это 1-ая формула $(A \rightarrow \overline{B}) = 0$, а 2-ая $((A \vee (B \Longleftrightarrow A)) \rightarrow C) = 0$. Если посчитать из первой формулы, то будет что $A = 1$, а $B = 1$, после этого подставим во 2-ую формулу и поймем, что, если у нас будет $C = 0$, все прекрасно выполниться.
\\
\\
Доказано!

\section{Упражнение}

Доказать равносильность без использования таблиц истинности:
\begin{enumerate}
\item[1)] $((((A \rightarrow B) \rightarrow \overline{A}) \rightarrow \overline{B}) \rightarrow \overline{C}) \rightarrow C = C$
\end{enumerate}
Разложим все по формуле: $(A \rightarrow B = \overline{A} \vee B)$\\
1. $\overline{A} \vee B \rightarrow \overline{A}$ \\
2. $(A \wedge \overline{B}) \vee \overline{A} \rightarrow \overline{B}$\\
3. $(\overline{A} \vee B) \wedge (A \vee \overline{B}) \rightarrow \overline{C}$\\
4. $(A \wedge \overline{B}) \vee (\overline{A} \wedge B) \vee \overline{C} \rightarrow C$\\
5. $(\overline{A} \vee B) \wedge (A \vee \overline{B}) \wedge C \vee C = C$\\
6. $1 \wedge C \vee C = C$\\
7. $C = C$

\begin{enumerate}
\item[2)] $(\overline{(A \wedge B \rightarrow C)} \rightarrow \overline{A \wedge C}) \rightarrow (A \wedge B \rightarrow \overline{A \wedge (B \rightarrow C )}) = \overline{A} \vee \overline{B} \vee \overline{C}$
\end{enumerate}

\section{Упражнение}

Доказать, что ФЛВ является тавтологиями:
\begin{enumerate}
\item[1)] $A \rightarrow A \vee B$
\end{enumerate}
Способ № 3: (тавтология)\\
$A \rightarrow A \vee B = 1$
$\overline{A} \vee A \vee B = 1$\\
$1 \vee B = 1$\\
Доказано!
\begin{enumerate}
\item[2)] $(A \rightarrow B) \Longleftrightarrow (\overline{B} \rightarrow \overline{A}) $
\end{enumerate}
Способ № 3: (тавтология)\\
$(A \rightarrow B) \Longleftrightarrow (\overline{B} \rightarrow \overline{A}) = 1$\\
$(\overline{A} \vee B) \Longleftrightarrow (B \vee \overline{A}) = 1$\\
Получилось две одинаковых формулы - формула тавтология.\\
Доказано!
\begin{enumerate}
\item[3)] $(A \rightarrow B) \wedge (B \rightarrow C) \rightarrow (A \rightarrow C)$
\end{enumerate}
Способ № 2: (таблица истинности)\\
\begin{tabular}{ | c | c | c | c | c | c | c| c | } % с - колонка с текстом по центру
$A$ & $B$ & $C$ & $A \rightarrow B$ & $B \rightarrow C$ & $A \rightarrow C$ & $(1)$ & $(2)$\\
\hline
0 & 0 & 0 & 1 & 1 & 1 & 1 & 1\\
0 & 0 & 1 & 1 & 1 & 1 & 1 & 1\\
0 & 1 & 0 & 1 & 0 & 1 & 0 & 1\\
0 & 1 & 1 & 1 & 1 & 1 & 1 & 1\\
1 & 0 & 0 & 0 & 1 & 0 & 0 & 1\\
1 & 0 & 1 & 0 & 1 & 1 & 0 & 1\\
1 & 1 & 0 & 1 & 0 & 0 & 0 & 1\\
1 & 1 & 1 & 1 & 1 & 1 & 1 & 1\\
\hline
\end{tabular} \\
Из таблицы истинности видим: конечный результат всегда будет равен 1 - формула тавтология.
\begin{enumerate}
\item[4)] $(A \vee B) \wedge ((A \rightarrow C) \wedge (B \rightarrow C)) \rightarrow C$
\end{enumerate}
Способ № 1: (от противного)\\
Допустим что наша формула равна 0, то получится, что $(A \vee B) \wedge ((A \rightarrow C) \wedge (B \rightarrow C)) = 1$, а $C = 0$. Если подставим C во все маленькие формулы то значения $B = 0, A = 0$, но при этом не выполняется условие $A \vee B$ значит формула тавтология.\\
\begin{enumerate}
\item[5)] $(A \Longleftrightarrow B) \wedge (B \Longleftrightarrow C) \rightarrow (A \rightarrow C)$
\end{enumerate}
Способ № 2: (таблица истинности)\\
\begin{tabular}{ | c | c | c | c | c | c | c| c | } % с - колонка с текстом по центру
$A$ & $B$ & $C$ & $A \Longleftrightarrow B$ & $B \Longleftrightarrow C$ & $A \rightarrow C$ & $(1)$ & $(2)$\\ \hline
0 & 0 & 0 & 1 & 1 & 1 & 1 & 1\\
0 & 0 & 1 & 1 & 0 & 1 & 0 & 1\\
0 & 1 & 0 & 0 & 0 & 1 & 0 & 1\\
0 & 1 & 1 & 0 & 1 & 1 & 0 & 1\\
1 & 0 & 0 & 0 & 1 & 0 & 0 & 1\\
1 & 0 & 1 & 0 & 0 & 1 & 0 & 1\\
1 & 1 & 0 & 1 & 0 & 0 & 0 & 1\\
1 & 1 & 1 & 1 & 1 & 1 & 1 & 1\\
\hline
\end{tabular} \\
Из таблицы истинности видим: конечный результат всегда будет равен 1 - формула тавтология.
\begin{enumerate}
\item[6)] $(A \rightarrow B) \rightarrow ((B \rightarrow C) \rightarrow (A \rightarrow C))$
\end{enumerate}
Способ № 1: (от противного)\\
Допустим что наша формула равна 0, то получится, что $(A \rightarrow B) = 1$, а $((B \rightarrow C) \rightarrow (A \rightarrow C)) = 0$. Если решим $((B \rightarrow C) \rightarrow (A \rightarrow C)) = 0$ то значения $A = 1, C = 0$. Подставим в формулу и получим, что $B = 1$, но при этом не выполняется условие $(B \rightarrow C)$ значит формула тавтология.\\

\section{Упражнение}
Проверить справедливость логических следований:(тремя способами)
\begin{enumerate}
\item[1)] $A \rightarrow B,D , D \rightarrow \overline{C},C \vee \overline{B} |= A \rightarrow \overline{D}$
\end{enumerate}
$(A \rightarrow B) \wedge D \wedge (D \rightarrow \overline{C}) \wedge (C \vee \overline{B}) \rightarrow (A \rightarrow \overline{D})$\\
Способ № 1: (от противного)\\
Допустим что наша формула равна 0, то получится, что $(A \rightarrow B) \wedge D \wedge (D \rightarrow \overline{C}) \wedge (C \vee \overline{B}) = 1$, а $(A \rightarrow \overline{D}) = 0$. Если решим все это, то получим значения $A = 1, B = 0, C = 0, D = 1$, но при этом не выполняется условие $(A \rightarrow B) = 1$ - формула тавтология.\\

\begin{enumerate}
\item[2)] $(A \rightarrow B) \rightarrow C$ |$=(A \rightarrow B) \rightarrow (A \rightarrow C)$
\end{enumerate}
$((A \rightarrow B) \rightarrow C) \rightarrow ((A \rightarrow B) \rightarrow (A \rightarrow C))$\\
Способ № 1: (от противного)\\
Допустим что наша формула равна 0, то получится, что $((A \rightarrow B) \rightarrow C) = 1$, а $((A \rightarrow B) \rightarrow (A \rightarrow C)) = 0$. Если решим все это, то получим значения $A = 1, B = 1, C = 0$, но при этом не выполняется условие $((A \rightarrow B) \rightarrow C) = 1$ - формула тавтология.\\
Способ № 2: (таблица истинности)\\
\begin{tabular}{ | c | c | c | c | c | c | c| c | } % с - колонка с текстом по центру
$A$ & $B$ & $C$ & $A \rightarrow B$ & $A \rightarrow C$ & $(1)$ & $(2)$ & $(3)$\\ \hline
0 & 0 & 0 & 1 & 1 & 0 & 1 & 1\\
0 & 0 & 1 & 1 & 1 & 1 & 1 & 1\\
0 & 1 & 0 & 1 & 1 & 0 & 1 & 1\\
0 & 1 & 1 & 1 & 1 & 1 & 1 & 1\\
1 & 0 & 0 & 0 & 0 & 1 & 1 & 1\\
1 & 0 & 1 & 0 & 1 & 1 & 1 & 1\\
1 & 1 & 0 & 1 & 0 & 0 & 0 & 1\\
1 & 1 & 1 & 1 & 1 & 1 & 1 & 1\\
\hline
\end{tabular} \\
Из таблицы истинности видим: конечный результат всегда будет равен 1 - формула тавтология.

\begin{enumerate}
\item[3)] $A \rightarrow B, B \rightarrow C$ |$= A \rightarrow C$
\end{enumerate}
$(A \rightarrow B) \wedge
(B \rightarrow C) \rightarrow (A \rightarrow C)$\\
Способ № 1: (от противного)\\
Допустим что наша формула равна 0, то получится, что $(A \rightarrow B) \wedge (B \rightarrow C) = 1$, а $(A \rightarrow C) = 0$. Если решим все это, то получим значения $A = 1, B = 1, C = 0$, но при этом не выполняется условие $(B \rightarrow C) = 1$ - формула тавтология.\\
Способ № 2: (таблица истинности)\\
\begin{tabular}{ | c | c | c | c | c | c | c| c | } % с - колонка с текстом по центру
$A$ & $B$ & $C$ & $A \rightarrow B$ & $B \rightarrow C$ & $A \rightarrow C$ & $(1)$ & $(2)$\\ \hline
0 & 0 & 0 & 1 & 1 & 1 & 1 & 1\\
0 & 0 & 1 & 1 & 1 & 1 & 1 & 1\\
0 & 1 & 0 & 1 & 0 & 1 & 0 & 1\\
0 & 1 & 1 & 1 & 1 & 1 & 1 & 1\\
1 & 0 & 0 & 0 & 1 & 0 & 0 & 1\\
1 & 0 & 1 & 0 & 1 & 1 & 0 & 1\\
1 & 1 & 0 & 1 & 0 & 0 & 0 & 1\\
1 & 1 & 1 & 1 & 1 & 1 & 1 & 1\\
\hline
\end{tabular} \\
Из таблицы истинности видим: конечный результат всегда будет равен 1 - формула тавтология.
\end{document}\documentclass{article}
\usepackage[utf8]{inputenc}
\usepackage{hyperref} % для гиперссылок
\usepackage[russian]{babel} % для русского языка
\usepackage{mathtools} % для математических выражений

\title{Дискретная математика \\ Лабораторная работа 2}

\begin{document}

\maketitle

\section{Упражнение}

Решить уравнения путем рассуждения:

\textbf{Выполнение}:
\begin{enumerate}
\item[1)] \overline{(A \Longleftrightarrow B)} \wedge \overline{(A \Longleftrightarrow C)} \rightarrow \overline{(A\Longleftrightarrow B \wedge D)} = 0
\end{enumerate}\\
\\
Чтобы это выполнялось 1-ая часть $(\overline{(A \Longleftrightarrow B)} \wedge \overline{(A \Longleftrightarrow C)})$ должна равняться 1, а 2-ая $(\overline{(A\Longleftrightarrow B \wedge D)})$ равняться 0.
Тогда А будет равняться 0, а $B = C = 1$. Тогда D будет 0 т.к. $B = 1$. \\
\\
Доказано!
\begin{enumerate}
\item[2)] (A \rightarrow C) \wedge \overline{((B \rightarrow C) \rightarrow ((A \vee B) \rightarrow C))} = 1
\end{enumerate}
\\
\\
Чтобы это выполнялось должно быть 1-ая формула $(A \rightarrow C) = 1$, 2-ая $(\overline{((B \rightarrow C) \rightarrow ((A \vee B) \rightarrow C))}) = 0$. Разложим на мелкие кусочки: $1 - (A \rightarrow C), 2 - (B \rightarrow C), 3 - ((A \vee B) \rightarrow C)$. При этом каждая формула должна соответствовать значению:\\
$(A \rightarrow C) = 1$ \\
$(B \rightarrow C) = 1$ \\
$((A \vee B) \rightarrow C) = 0$ \\
При этом находим, что $C = 0$, $A = 0$, $B = 0$. Теперь подставим все под нашу формулу:
$(0 \rightarrow 0) \wedge \overline{(0 \rightarrow 0) \rightarrow ((0 \vee 0) \rightarrow 0))} = 1$.\\
Выходит:\\
$0 = 1$ \\
\\
НЕ доказано!

\begin{enumerate}
\item[3)] \overline{(A \rightarrow \overline{B})} \rightarrow \overline{((A \vee (B \Longleftrightarrow A)) \rightarrow C)} = 1
\end{enumerate} \\
\\
Чтобы выполнялось это 1-ая формула $(A \rightarrow \overline{B}) = 0$, а 2-ая $((A \vee (B \Longleftrightarrow A)) \rightarrow C) = 0$. Если посчитать из первой формулы, то будет что $A = 1$, а $B = 1$, после этого подставим во 2-ую формулу и поймем, что, если у нас будет $C = 0$, все прекрасно выполниться.
\\
\\
Доказано!

\section{Упражнение}

Доказать равносильность без использования таблиц истинности:
\begin{enumerate}
\item[1)] $((((A \rightarrow B) \rightarrow \overline{A}) \rightarrow \overline{B}) \rightarrow \overline{C}) \rightarrow C = C$
\end{enumerate}
Разложим все по формуле: $(A \rightarrow B = \overline{A} \vee B)$\\
1. $\overline{A} \vee B \rightarrow \overline{A}$ \\
2. $(A \wedge \overline{B}) \vee \overline{A} \rightarrow \overline{B}$\\
3. $(\overline{A} \vee B) \wedge (A \vee \overline{B}) \rightarrow \overline{C}$\\
4. $(A \wedge \overline{B}) \vee (\overline{A} \wedge B) \vee \overline{C} \rightarrow C$\\
5. $(\overline{A} \vee B) \wedge (A \vee \overline{B}) \wedge C \vee C = C$\\
6. $1 \wedge C \vee C = C$\\
7. $C = C$

\begin{enumerate}
\item[2)] $(\overline{(A \wedge B \rightarrow C)} \rightarrow \overline{A \wedge C}) \rightarrow (A \wedge B \rightarrow \overline{A \wedge (B \rightarrow C )}) = \overline{A} \vee \overline{B} \vee
\overline{C}$
\end{enumerate}

\section{Упражнение}

Доказать, что ФЛВ является тавтологиями:
\begin{enumerate}
\item[1)] $A \rightarrow A \vee B$
\end{enumerate}
Способ № 3: (тавтология)\\
$A \rightarrow A \vee B = 1$
$\overline{A} \vee A \vee B = 1$\\
$1 \vee B = 1$\\
Доказано!
\begin{enumerate}
\item[2)] $(A \rightarrow B) \Longleftrightarrow (\overline{B} \rightarrow \overline{A}) $
\end{enumerate}
Способ № 3: (тавтология)\\
$(A \rightarrow B) \Longleftrightarrow (\overline{B} \rightarrow \overline{A}) = 1$\\
$(\overline{A} \vee B) \Longleftrightarrow (B \vee \overline{A}) = 1$\\
Получилось две одинаковых формулы - формула тавтология.\\
Доказано!
\begin{enumerate}
\item[3)] $(A \rightarrow B) \wedge (B \rightarrow C) \rightarrow (A \rightarrow C)$
\end{enumerate}
Способ № 2: (таблица истинности)\\
\begin{tabular}{ | c | c | c | c | c | c | c| c | } % с - колонка с текстом по центру
$A$ & $B$ & $C$ & $A \rightarrow B$ & $B \rightarrow C$ & $A \rightarrow C$ & $(1)$ & $(2)$\\ \hline
0 & 0 & 0 & 1 & 1 & 1 & 1 & 1\\
0 & 0 & 1 & 1 & 1 & 1 & 1 & 1\\
0 & 1 & 0 & 1 & 0 & 1 & 0 & 1\\
0 & 1 & 1 & 1 & 1 & 1 & 1 & 1\\
1 & 0 & 0 & 0 & 1 & 0 & 0 & 1\\
1 & 0 & 1 & 0 & 1 & 1 & 0 & 1\\
1 & 1 & 0 & 1 & 0 & 0 & 0 & 1\\
1 & 1 & 1 & 1 & 1 & 1 & 1 & 1\\
\hline
\end{tabular} \\
Из таблицы истинности видим: конечный результат всегда будет равен 1 - формула тавтология.
\begin{enumerate}
\item[4)] $(A \vee B) \wedge ((A \rightarrow C) \wedge (B \rightarrow C)) \rightarrow C$
\end{enumerate}
Способ № 1: (от противного)\\
Допустим что наша формула равна 0, то получится, что $(A \vee B) \wedge ((A \rightarrow C) \wedge (B \rightarrow C)) = 1$, а $C = 0$. Если подставим C во все маленькие формулы то значения $B = 0, A = 0$, но при этом не выполняется условие $A \vee B$ значит формула тавтология.\\
\begin{enumerate}
\item[5)] $(A \Longleftrightarrow B) \wedge (B \Longleftrightarrow C) \rightarrow (A \rightarrow C)$
\end{enumerate}
Способ № 2: (таблица истинности)\\
\begin{tabular}{ | c | c | c | c | c | c | c| c | } % с - колонка с текстом по центру
$A$ & $B$ & $C$ & $A \Longleftrightarrow B$ & $B \Longleftrightarrow C$ & $A \rightarrow C$ & $(1)$ & $(2)$\\ \hline
0 & 0 & 0 & 1 & 1 & 1 & 1 & 1\\
0 & 0 & 1 & 1 & 0 & 1 & 0 & 1\\
0 & 1 & 0 & 0 & 0 & 1 & 0 & 1\\
0 & 1 & 1 & 0 & 1 & 1 & 0 & 1\\
1 & 0 & 0 & 0 & 1 & 0 & 0 & 1\\
1 & 0 & 1 & 0 & 0 & 1 & 0 & 1\\
1 & 1 & 0 & 1 & 0 & 0 & 0 & 1\\
1 & 1 & 1 & 1 & 1 & 1 & 1 & 1\\
\hline
\end{tabular} \\
Из таблицы истинности видим: конечный результат всегда будет равен 1 - формула тавтология.
\begin{enumerate}
\item[6)] $(A \rightarrow B) \rightarrow ((B \rightarrow C) \rightarrow (A \rightarrow C))$
\end{enumerate}
Способ № 1: (от противного)\\
Допустим что наша формула равна 0, то получится, что $(A \rightarrow B) = 1$, а $((B \rightarrow C) \rightarrow (A \rightarrow C)) = 0$. Если решим $((B \rightarrow C) \rightarrow (A \rightarrow C)) = 0$ то значения $A = 1, C = 0$. Подставим в формулу и получим, что $B = 1$, но при этом не выполняется условие $(B \rightarrow C)$ значит формула тавтология.\\

\section{Упражнение}
Проверить справедливость логических следований:(тремя способами)
\begin{enumerate}
\item[1)] $A \rightarrow B,D , D \rightarrow \overline{C},C \vee \overline{B} |= A \rightarrow \overline{D}$
\end{enumerate}
$(A \rightarrow B) \wedge D \wedge (D \rightarrow \overline{C}) \wedge (C \vee \overline{B}) \rightarrow (A \rightarrow \overline{D})$\\
Способ № 1: (от противного)\\
Допустим что наша формула равна 0, то получится, что $(A \rightarrow B) \wedge D \wedge (D \rightarrow \overline{C}) \wedge (C \vee \overline{B}) = 1$, а $(A \rightarrow \overline{D}) = 0$. Если решим все это, то получим значения $A = 1, B = 0, C = 0, D = 1$, но при этом не выполняется условие $(A \rightarrow B) = 1$ - формула тавтология.\\

\begin{enumerate}
\item[2)] $(A \rightarrow B) \rightarrow C$ |$=(A \rightarrow B) \rightarrow (A \rightarrow C)$
\end{enumerate}
$((A \rightarrow B) \rightarrow C) \rightarrow ((A \rightarrow B) \rightarrow (A \rightarrow C))$\\
Способ № 1: (от противного)\\
Допустим что наша формула равна
0, то получится, что $((A \rightarrow B) \rightarrow C) = 1$, а $((A \rightarrow B) \rightarrow (A \rightarrow C)) = 0$. Если решим все это, то получим значения $A = 1, B = 1, C = 0$, но при этом не выполняется условие $((A \rightarrow B) \rightarrow C) = 1$ - формула тавтология.\\
Способ № 2: (таблица истинности)\\
\begin{tabular}{ | c | c | c | c | c | c | c| c | } % с - колонка с текстом по центру
$A$ & $B$ & $C$ & $A \rightarrow B$ & $A \rightarrow C$ & $(1)$ & $(2)$ & $(3)$\\ \hline
0 & 0 & 0 & 1 & 1 & 0 & 1 & 1\\
0 & 0 & 1 & 1 & 1 & 1 & 1 & 1\\
0 & 1 & 0 & 1 & 1 & 0 & 1 & 1\\
0 & 1 & 1 & 1 & 1 & 1 & 1 & 1\\
1 & 0 & 0 & 0 & 0 & 1 & 1 & 1\\
1 & 0 & 1 & 0 & 1 & 1 & 1 & 1\\
1 & 1 & 0 & 1 & 0 & 0 & 0 & 1\\
1 & 1 & 1 & 1 & 1 & 1 & 1 & 1\\
\hline
\end{tabular} \\
Из таблицы истинности видим: конечный результат всегда будет равен 1 - формула тавтология.

\begin{enumerate}
\item[3)] $A \rightarrow B, B \rightarrow C$ |$= A \rightarrow C$
\end{enumerate}
$(A \rightarrow B) \wedge (B \rightarrow C) \rightarrow (A \rightarrow C)$\\
Способ № 1: (от противного)\\
Допустим что наша формула равна 0, то получится, что $(A \rightarrow B) \wedge (B \rightarrow C) = 1$, а $(A \rightarrow C) = 0$. Если решим все это, то получим значения $A = 1, B = 1, C = 0$, но при этом не выполняется условие $(B \rightarrow C) = 1$ - формула тавтология.\\
Способ № 2: (таблица истинности)\\
\begin{tabular}{ | c | c | c | c | c | c | c| c | } % с - колонка с текстом по центру
$A$ & $B$ & $C$ & $A \rightarrow B$ & $B \rightarrow C$ & $A \rightarrow C$ & $(1)$ & $(2)$\\ \hline
0 & 0 & 0 & 1 & 1 & 1 & 1 & 1\\
0 & 0 & 1 & 1 & 1 & 1 & 1 & 1\\
0 & 1 & 0 & 1 & 0 & 1 & 0 & 1\\
0 & 1 & 1 & 1 & 1 & 1 & 1 & 1\\
1 & 0 & 0 & 0 & 1 & 0 & 0 & 1\\
1 & 0 & 1 & 0 & 1 & 1 & 0 & 1\\
1 & 1 & 0 & 1 & 0 & 0 & 0 & 1\\
1 & 1 & 1 & 1 & 1 & 1 & 1 & 1\\
\hline
\end{tabular} \\
Из таблицы истинности видим: конечный результат всегда будет равен 1 - формула тавтология.
\end{document}\documentclass{article}
\usepackage[utf8]{inputenc}
\usepackage{hyperref} % для гиперссылок
\usepackage[russian]{babel} % для русского языка
\usepackage{mathtools} % для математических выражений

\title{Дискретная математика \\ Лабораторная работа 2}

\begin{document}

\maketitle

\section{Упражнение}

Решить уравнения путем рассуждения:

\textbf{Выполнение}:
\begin{enumerate}
\item[1)] \overline{(A \Longleftrightarrow B)} \wedge \overline{(A \Longleftrightarrow C)} \rightarrow \overline{(A\Longleftrightarrow B \wedge D)} = 0
\end{enumerate}\\
\\
Чтобы это выполнялось 1-ая часть $(\overline{(A \Longleftrightarrow B)} \wedge \overline{(A \Longleftrightarrow C)})$ должна равняться 1, а 2-ая $(\overline{(A\Longleftrightarrow B \wedge D)})$ равняться 0.
Тогда А будет равняться 0, а $B = C = 1$. Тогда D будет 0 т.к. $B = 1$. \\
\\
Доказано!
\begin{enumerate}
\item[2)] (A \rightarrow C) \wedge \overline{((B \rightarrow C) \rightarrow ((A \vee B) \rightarrow C))} = 1
\end{enumerate}
\\
\\
Чтобы это выполнялось должно быть 1-ая формула $(A \rightarrow C) = 1$, 2-ая $(\overline{((B \rightarrow C) \rightarrow ((A \vee B) \rightarrow C))}) = 0$. Разложим на мелкие кусочки: $1 - (A \rightarrow C), 2 - (B \rightarrow C), 3 - ((A \vee B) \rightarrow C)$. При этом каждая формула должна соответствовать значению:\\
$(A \rightarrow C) = 1$ \\
$(B \rightarrow C) = 1$ \\
$((A \vee B) \rightarrow C) = 0$ \\
При этом находим, что $C = 0$, $A = 0$, $B = 0$. Теперь подставим все под нашу формулу:
$(0 \rightarrow 0) \wedge \overline{(0 \rightarrow 0) \rightarrow ((0 \vee 0) \rightarrow 0))} = 1$.\\
Выходит:\\
$0 = 1$ \\
\\
НЕ доказано!

\begin{enumerate}
\item[3)] \overline{(A \rightarrow \overline{B})} \rightarrow \overline{((A \vee (B \Longleftrightarrow A)) \rightarrow C)} = 1
\end{enumerate} \\
\\
Чтобы выполнялось это 1-ая формула $(A \rightarrow \overline{B}) = 0$, а 2-ая $((A \vee (B \Longleftrightarrow A)) \rightarrow C) = 0$. Если посчитать из первой формулы, то будет что $A = 1$, а $B = 1$, после этого подставим во 2-ую формулу и поймем, что, если у нас будет $C = 0$, все прекрасно
выполниться.
\\
\\
Доказано!

\section{Упражнение}

Доказать равносильность без использования таблиц истинности:
\begin{enumerate}
\item[1)] $((((A \rightarrow B) \rightarrow \overline{A}) \rightarrow \overline{B}) \rightarrow \overline{C}) \rightarrow C = C$
\end{enumerate}
Разложим все по формуле: $(A \rightarrow B = \overline{A} \vee B)$\\
1. $\overline{A} \vee B \rightarrow \overline{A}$ \\
2. $(A \wedge \overline{B}) \vee \overline{A} \rightarrow \overline{B}$\\
3. $(\overline{A} \vee B) \wedge (A \vee \overline{B}) \rightarrow \overline{C}$\\
4. $(A \wedge \overline{B}) \vee (\overline{A} \wedge B) \vee \overline{C} \rightarrow C$\\
5. $(\overline{A} \vee B) \wedge (A \vee \overline{B}) \wedge C \vee C = C$\\
6. $1 \wedge C \vee C = C$\\
7. $C = C$

\begin{enumerate}
\item[2)] $(\overline{(A \wedge B \rightarrow C)} \rightarrow \overline{A \wedge C}) \rightarrow (A \wedge B \rightarrow \overline{A \wedge (B \rightarrow C )}) = \overline{A} \vee \overline{B} \vee \overline{C}$
\end{enumerate}

\section{Упражнение}

Доказать, что ФЛВ является тавтологиями:
\begin{enumerate}
\item[1)] $A \rightarrow A \vee B$
\end{enumerate}
Способ № 3: (тавтология)\\
$A \rightarrow A \vee B = 1$
$\overline{A} \vee A \vee B = 1$\\
$1 \vee B = 1$\\
Доказано!
\begin{enumerate}
\item[2)] $(A \rightarrow B) \Longleftrightarrow (\overline{B} \rightarrow \overline{A}) $
\end{enumerate}
Способ № 3: (тавтология)\\
$(A \rightarrow B) \Longleftrightarrow (\overline{B} \rightarrow \overline{A}) = 1$\\
$(\overline{A} \vee B) \Longleftrightarrow (B \vee \overline{A}) = 1$\\
Получилось две одинаковых формулы - формула тавтология.\\
Доказано!
\begin{enumerate}
\item[3)] $(A \rightarrow B) \wedge (B \rightarrow C) \rightarrow (A \rightarrow C)$
\end{enumerate}
Способ № 2: (таблица истинности)\\
\begin{tabular}{ | c | c | c | c | c | c | c| c | } % с - колонка с текстом по центру
$A$ & $B$ & $C$ & $A \rightarrow B$ & $B \rightarrow C$ & $A \rightarrow C$ & $(1)$ & $(2)$\\ \hline
0 & 0 & 0 & 1 & 1 & 1 & 1 & 1\\
0 & 0 & 1 & 1 & 1 & 1 & 1 & 1\\
0 & 1 & 0 & 1 & 0 & 1 & 0 & 1\\
0 & 1 & 1 & 1 & 1 & 1 & 1 & 1\\
1 & 0 & 0 & 0 & 1 & 0 & 0 & 1\\
1 & 0 & 1 & 0 & 1 & 1 & 0 & 1\\
1 & 1 & 0 & 1 & 0 & 0 & 0 & 1\\
1 & 1 & 1 & 1 & 1 & 1 & 1 & 1\\
\hline
\end{tabular} \\
Из таблицы истинности видим: конечный результат всегда будет равен 1 - формула тавтология.
\begin{enumerate}
\item[4)] $(A \vee B) \wedge ((A \rightarrow C) \wedge (B \rightarrow C)) \rightarrow C$
\end{enumerate}
Способ № 1: (от противного)\\
Допустим что наша формула равна 0, то получится, что $(A \vee B) \wedge ((A \rightarrow C) \wedge (B \rightarrow C)) = 1$, а $C = 0$. Если подставим C во все маленькие формулы то значения $B = 0, A = 0$, но при этом не выполняется условие $A \vee B$ значит формула тавтология.\\
\begin{enumerate}
\item[5)] $(A \Longleftrightarrow B) \wedge (B \Longleftrightarrow C) \rightarrow (A \rightarrow C)$
\end{enumerate}
Способ № 2: (таблица истинности)\\
\begin{tabular}{ | c | c | c | c | c | c | c| c | } % с - колонка с текстом по центру
$A$ & $B$ & $C$ & $A \Longleftrightarrow B$ & $B \Longleftrightarrow C$ & $A \rightarrow C$ & $(1)$ & $(2)$\\ \hline
0 & 0 & 0 & 1 & 1 & 1 & 1 & 1\\
0 & 0 & 1 & 1 & 0 & 1 & 0 & 1\\
0 & 1 & 0 & 0 & 0 & 1 & 0 & 1\\
0 & 1 & 1 & 0 & 1 & 1 & 0 & 1\\
1 & 0 & 0 & 0 & 1 & 0 & 0 & 1\\
1 & 0 & 1 & 0 & 0 & 1 & 0 & 1\\
1 & 1 & 0 & 1 & 0 & 0 & 0 & 1\\
1 & 1 & 1 & 1 & 1 & 1 & 1 & 1\\
\hline
\end{tabular} \\
Из таблицы истинности видим: конечный результат всегда будет равен 1 - формула тавтология.
\begin{enumerate}
\item[6)] $(A \rightarrow B) \rightarrow ((B \rightarrow C) \rightarrow (A \rightarrow C))$
\end{enumerate}
Способ № 1: (от противного)\\
Допустим что наша формула равна 0, то получится, что $(A \rightarrow B) = 1$, а $((B \rightarrow C) \rightarrow (A \rightarrow C)) = 0$. Если решим $((B \rightarrow C) \rightarrow (A \rightarrow C)) = 0$ то значения $A = 1, C = 0$. Подставим в формулу и получим, что $B = 1$, но при этом не выполняется условие $(B \rightarrow C)$ значит формула тавтология.\\

\section{Упражнение}
Проверить справедливость
логических следований:(тремя способами)
\begin{enumerate}
\item[1)] $A \rightarrow B,D , D \rightarrow \overline{C},C \vee \overline{B} |= A \rightarrow \overline{D}$
\end{enumerate}
$(A \rightarrow B) \wedge D \wedge (D \rightarrow \overline{C}) \wedge (C \vee \overline{B}) \rightarrow (A \rightarrow \overline{D})$\\
Способ № 1: (от противного)\\
Допустим что наша формула равна 0, то получится, что $(A \rightarrow B) \wedge D \wedge (D \rightarrow \overline{C}) \wedge (C \vee \overline{B}) = 1$, а $(A \rightarrow \overline{D}) = 0$. Если решим все это, то получим значения $A = 1, B = 0, C = 0, D = 1$, но при этом не выполняется условие $(A \rightarrow B) = 1$ - формула тавтология.\\

\begin{enumerate}
\item[2)] $(A \rightarrow B) \rightarrow C$ |$=(A \rightarrow B) \rightarrow (A \rightarrow C)$
\end{enumerate}
$((A \rightarrow B) \rightarrow C) \rightarrow ((A \rightarrow B) \rightarrow (A \rightarrow C))$\\
Способ № 1: (от противного)\\
Допустим что наша формула равна 0, то получится, что $((A \rightarrow B) \rightarrow C) = 1$, а $((A \rightarrow B) \rightarrow (A \rightarrow C)) = 0$. Если решим все это, то получим значения $A = 1, B = 1, C = 0$, но при этом не выполняется условие $((A \rightarrow B) \rightarrow C) = 1$ - формула тавтология.\\
Способ № 2: (таблица истинности)\\
\begin{tabular}{ | c | c | c | c | c | c | c| c | } % с - колонка с текстом по центру
$A$ & $B$ & $C$ & $A \rightarrow B$ & $A \rightarrow C$ & $(1)$ & $(2)$ & $(3)$\\ \hline
0 & 0 & 0 & 1 & 1 & 0 & 1 & 1\\
0 & 0 & 1 & 1 & 1 & 1 & 1 & 1\\
0 & 1 & 0 & 1 & 1 & 0 & 1 & 1\\
0 & 1 & 1 & 1 & 1 & 1 & 1 & 1\\
1 & 0 & 0 & 0 & 0 & 1 & 1 & 1\\
1 & 0 & 1 & 0 & 1 & 1 & 1 & 1\\
1 & 1 & 0 & 1 & 0 & 0 & 0 & 1\\
1 & 1 & 1 & 1 & 1 & 1 & 1 & 1\\
\hline
\end{tabular} \\
Из таблицы истинности видим: конечный результат всегда будет равен 1 - формула тавтология.

\begin{enumerate}
\item[3)] $A \rightarrow B, B \rightarrow C$ |$= A \rightarrow C$
\end{enumerate}
$(A \rightarrow B) \wedge (B \rightarrow C) \rightarrow (A \rightarrow C)$\\
Способ № 1: (от противного)\\
Допустим что наша формула равна 0, то получится, что $(A \rightarrow B) \wedge (B \rightarrow C) = 1$, а $(A \rightarrow C) = 0$. Если решим все это, то получим значения $A = 1, B = 1, C = 0$, но при этом не выполняется условие $(B \rightarrow C) = 1$ - формула тавтология.\\
Способ № 2: (таблица истинности)\\
\begin{tabular}{ | c | c | c | c | c | c | c| c | } % с - колонка с текстом по центру
$A$ & $B$ & $C$ & $A \rightarrow B$ & $B \rightarrow C$ & $A \rightarrow C$ & $(1)$ & $(2)$\\ \hline
0 & 0 & 0 & 1 & 1 & 1 & 1 & 1\\
0 & 0 & 1 & 1 & 1 & 1 & 1 & 1\\
0 & 1 & 0 & 1 & 0 & 1 & 0 & 1\\
0 & 1 & 1 & 1 & 1 & 1 & 1 & 1\\
1 & 0 & 0 & 0 & 1 & 0 & 0 & 1\\
1 & 0 & 1 & 0 & 1 & 1 & 0 & 1\\
1 & 1 & 0 & 1 & 0 & 0 & 0 & 1\\
1 & 1 & 1 & 1 & 1 & 1 & 1 & 1\\
\hline
\end{tabular} \\
Из таблицы истинности видим: конечный результат всегда будет равен 1 - формула тавтология.
\end{document}\documentclass{article}
\usepackage[utf8]{inputenc}
\usepackage{hyperref} % для гиперссылок
\usepackage[russian]{babel} % для русского языка
\usepackage{mathtools} % для математических выражений

\title{Дискретная математика \\ Лабораторная работа 2}

\begin{document}

\maketitle

\section{Упражнение}

Решить уравнения путем рассуждения:

\textbf{Выполнение}:
\begin{enumerate}
\item[1)] \overline{(A \Longleftrightarrow B)} \wedge \overline{(A \Longleftrightarrow C)} \rightarrow \overline{(A\Longleftrightarrow B \wedge D)} = 0
\end{enumerate}\\
\\
Чтобы это выполнялось 1-ая часть $(\overline{(A \Longleftrightarrow B)} \wedge \overline{(A \Longleftrightarrow C)})$ должна равняться 1, а 2-ая $(\overline{(A\Longleftrightarrow B \wedge D)})$ равняться 0.
Тогда А будет равняться 0, а $B = C = 1$. Тогда D будет 0 т.к. $B = 1$. \\
\\
Доказано!
\begin{enumerate}
\item[2)] (A \rightarrow C) \wedge \overline{((B \rightarrow C) \rightarrow ((A \vee B) \rightarrow C))} = 1
\end{enumerate}
\\
\\
Чтобы это выполнялось должно быть 1-ая формула $(A \rightarrow C) = 1$, 2-ая $(\overline{((B \rightarrow C) \rightarrow ((A \vee B) \rightarrow C))}) =
0$. Разложим на мелкие кусочки: $1 - (A \rightarrow C), 2 - (B \rightarrow C), 3 - ((A \vee B) \rightarrow C)$. При этом каждая формула должна соответствовать значению:\\
$(A \rightarrow C) = 1$ \\
$(B \rightarrow C) = 1$ \\
$((A \vee B) \rightarrow C) = 0$ \\
При этом находим, что $C = 0$, $A = 0$, $B = 0$. Теперь подставим все под нашу формулу:
$(0 \rightarrow 0) \wedge \overline{(0 \rightarrow 0) \rightarrow ((0 \vee 0) \rightarrow 0))} = 1$.\\
Выходит:\\
$0 = 1$ \\
\\
НЕ доказано!

\begin{enumerate}
\item[3)] \overline{(A \rightarrow \overline{B})} \rightarrow \overline{((A \vee (B \Longleftrightarrow A)) \rightarrow C)} = 1
\end{enumerate} \\
\\
Чтобы выполнялось это 1-ая формула $(A \rightarrow \overline{B}) = 0$, а 2-ая $((A \vee (B \Longleftrightarrow A)) \rightarrow C) = 0$. Если посчитать из первой формулы, то будет что $A = 1$, а $B = 1$, после этого подставим во 2-ую формулу и поймем, что, если у нас будет $C = 0$, все прекрасно выполниться.
\\
\\
Доказано!

\section{Упражнение}

Доказать равносильность без использования таблиц истинности:
\begin{enumerate}
\item[1)] $((((A \rightarrow B) \rightarrow \overline{A}) \rightarrow \overline{B}) \rightarrow \overline{C}) \rightarrow C = C$
\end{enumerate}
Разложим все по формуле: $(A \rightarrow B = \overline{A} \vee B)$\\
1. $\overline{A} \vee B \rightarrow \overline{A}$ \\
2. $(A \wedge \overline{B}) \vee \overline{A} \rightarrow \overline{B}$\\
3. $(\overline{A} \vee B) \wedge (A \vee \overline{B}) \rightarrow \overline{C}$\\
4. $(A \wedge \overline{B}) \vee (\overline{A} \wedge B) \vee \overline{C} \rightarrow C$\\
5. $(\overline{A} \vee B) \wedge (A \vee \overline{B}) \wedge C \vee C = C$\\
6. $1 \wedge C \vee C = C$\\
7. $C = C$

\begin{enumerate}
\item[2)] $(\overline{(A \wedge B \rightarrow C)} \rightarrow \overline{A \wedge C}) \rightarrow (A \wedge B \rightarrow \overline{A \wedge (B \rightarrow C )}) = \overline{A} \vee \overline{B} \vee \overline{C}$
\end{enumerate}

\section{Упражнение}

Доказать, что ФЛВ является тавтологиями:
\begin{enumerate}
\item[1)] $A \rightarrow A \vee B$
\end{enumerate}
Способ № 3: (тавтология)\\
$A \rightarrow A \vee B = 1$
$\overline{A} \vee A \vee B = 1$\\
$1 \vee B = 1$\\
Доказано!
\begin{enumerate}
\item[2)] $(A \rightarrow B) \Longleftrightarrow (\overline{B} \rightarrow \overline{A}) $
\end{enumerate}
Способ № 3: (тавтология)\\
$(A \rightarrow B) \Longleftrightarrow (\overline{B} \rightarrow \overline{A}) = 1$\\
$(\overline{A} \vee B) \Longleftrightarrow (B \vee \overline{A}) = 1$\\
Получилось две одинаковых формулы - формула тавтология.\\
Доказано!
\begin{enumerate}
\item[3)] $(A \rightarrow B) \wedge (B \rightarrow C) \rightarrow (A \rightarrow C)$
\end{enumerate}
Способ № 2: (таблица истинности)\\
\begin{tabular}{ | c | c | c | c | c | c | c| c | } % с - колонка с текстом по центру
$A$ & $B$ & $C$ & $A \rightarrow B$ & $B \rightarrow C$ & $A \rightarrow C$ & $(1)$ & $(2)$\\ \hline
0 & 0 & 0 & 1 & 1 & 1 & 1 & 1\\
0 & 0 & 1 & 1 & 1 & 1 & 1 & 1\\
0 & 1 & 0 & 1 & 0 & 1 & 0 & 1\\
0 & 1 & 1 & 1 & 1 & 1 & 1 & 1\\
1 & 0 & 0 & 0 & 1 & 0 & 0 & 1\\
1 & 0 & 1 & 0 & 1 & 1 & 0 & 1\\
1 & 1 & 0 & 1 & 0 & 0 & 0 & 1\\
1 & 1 & 1 & 1 & 1 & 1 & 1 & 1\\
\hline
\end{tabular} \\
Из таблицы истинности видим: конечный результат всегда будет равен 1 - формула тавтология.
\begin{enumerate}
\item[4)] $(A \vee B) \wedge ((A \rightarrow C) \wedge (B \rightarrow C)) \rightarrow C$
\end{enumerate}
Способ № 1: (от противного)\\
Допустим что наша формула равна 0, то получится, что $(A \vee B) \wedge ((A \rightarrow C) \wedge (B \rightarrow C)) = 1$, а $C = 0$. Если подставим C во все маленькие формулы то значения $B = 0, A = 0$, но при этом не выполняется условие $A \vee B$ значит формула тавтология.\\
\begin{enumerate}
\item[5)] $(A \Longleftrightarrow B) \wedge (B \Longleftrightarrow C) \rightarrow (A \rightarrow C)$
\end{enumerate}
Способ № 2: (таблица истинности)\\
\begin{tabular}{ | c | c | c | c | c | c | c| c | } % с - колонка с текстом по центру
$A$ & $B$ & $C$ & $A \Longleftrightarrow B$ & $B \Longleftrightarrow C$ & $A \rightarrow C$ &
$(1)$ & $(2)$\\ \hline
0 & 0 & 0 & 1 & 1 & 1 & 1 & 1\\
0 & 0 & 1 & 1 & 0 & 1 & 0 & 1\\
0 & 1 & 0 & 0 & 0 & 1 & 0 & 1\\
0 & 1 & 1 & 0 & 1 & 1 & 0 & 1\\
1 & 0 & 0 & 0 & 1 & 0 & 0 & 1\\
1 & 0 & 1 & 0 & 0 & 1 & 0 & 1\\
1 & 1 & 0 & 1 & 0 & 0 & 0 & 1\\
1 & 1 & 1 & 1 & 1 & 1 & 1 & 1\\
\hline
\end{tabular} \\
Из таблицы истинности видим: конечный результат всегда будет равен 1 - формула тавтология.
\begin{enumerate}
\item[6)] $(A \rightarrow B) \rightarrow ((B \rightarrow C) \rightarrow (A \rightarrow C))$
\end{enumerate}
Способ № 1: (от противного)\\
Допустим что наша формула равна 0, то получится, что $(A \rightarrow B) = 1$, а $((B \rightarrow C) \rightarrow (A \rightarrow C)) = 0$. Если решим $((B \rightarrow C) \rightarrow (A \rightarrow C)) = 0$ то значения $A = 1, C = 0$. Подставим в формулу и получим, что $B = 1$, но при этом не выполняется условие $(B \rightarrow C)$ значит формула тавтология.\\

\section{Упражнение}
Проверить справедливость логических следований:(тремя способами)
\begin{enumerate}
\item[1)] $A \rightarrow B,D , D \rightarrow \overline{C},C \vee \overline{B} |= A \rightarrow \overline{D}$
\end{enumerate}
$(A \rightarrow B) \wedge D \wedge (D \rightarrow \overline{C}) \wedge (C \vee \overline{B}) \rightarrow (A \rightarrow \overline{D})$\\
Способ № 1: (от противного)\\
Допустим что наша формула равна 0, то получится, что $(A \rightarrow B) \wedge D \wedge (D \rightarrow \overline{C}) \wedge (C \vee \overline{B}) = 1$, а $(A \rightarrow \overline{D}) = 0$. Если решим все это, то получим значения $A = 1, B = 0, C = 0, D = 1$, но при этом не выполняется условие $(A \rightarrow B) = 1$ - формула тавтология.\\

\begin{enumerate}
\item[2)] $(A \rightarrow B) \rightarrow C$ |$=(A \rightarrow B) \rightarrow (A \rightarrow C)$
\end{enumerate}
$((A \rightarrow B) \rightarrow C) \rightarrow ((A \rightarrow B) \rightarrow (A \rightarrow C))$\\
Способ № 1: (от противного)\\
Допустим что наша формула равна 0, то получится, что $((A \rightarrow B) \rightarrow C) = 1$, а $((A \rightarrow B) \rightarrow (A \rightarrow C)) = 0$. Если решим все это, то получим значения $A = 1, B = 1, C = 0$, но при этом не выполняется условие $((A \rightarrow B) \rightarrow C) = 1$ - формула тавтология.\\
Способ № 2: (таблица истинности)\\
\begin{tabular}{ | c | c | c | c | c | c | c| c | } % с - колонка с текстом по центру
$A$ & $B$ & $C$ & $A \rightarrow B$ & $A \rightarrow C$ & $(1)$ & $(2)$ & $(3)$\\ \hline
0 & 0 & 0 & 1 & 1 & 0 & 1 & 1\\
0 & 0 & 1 & 1 & 1 & 1 & 1 & 1\\
0 & 1 & 0 & 1 & 1 & 0 & 1 & 1\\
0 & 1 & 1 & 1 & 1 & 1 & 1 & 1\\
1 & 0 & 0 & 0 & 0 & 1 & 1 & 1\\
1 & 0 & 1 & 0 & 1 & 1 & 1 & 1\\
1 & 1 & 0 & 1 & 0 & 0 & 0 & 1\\
1 & 1 & 1 & 1 & 1 & 1 & 1 & 1\\
\hline
\end{tabular} \\
Из таблицы истинности видим: конечный результат всегда будет равен 1 - формула тавтология.

\begin{enumerate}
\item[3)] $A \rightarrow B, B \rightarrow C$ |$= A \rightarrow C$
\end{enumerate}
$(A \rightarrow B) \wedge (B \rightarrow C) \rightarrow (A \rightarrow C)$\\
Способ № 1: (от противного)\\
Допустим что наша формула равна 0, то получится, что $(A \rightarrow B) \wedge (B \rightarrow C) = 1$, а $(A \rightarrow C) = 0$. Если решим все это, то получим значения $A = 1, B = 1, C = 0$, но при этом не выполняется условие $(B \rightarrow C) = 1$ - формула тавтология.\\
Способ № 2: (таблица истинности)\\
\begin{tabular}{ | c | c | c | c | c | c | c| c | } % с - колонка с текстом по центру
$A$ & $B$ & $C$ & $A \rightarrow B$ & $B \rightarrow C$ & $A \rightarrow C$ & $(1)$ & $(2)$\\ \hline
0 & 0 & 0 & 1 & 1 & 1 & 1 & 1\\
0 & 0 & 1 & 1 & 1 & 1 & 1 & 1\\
0 & 1 & 0 & 1 & 0 & 1 & 0 & 1\\
0 & 1 & 1 & 1 & 1 & 1 & 1 & 1\\
1 & 0 & 0 & 0 & 1 & 0 & 0 & 1\\
1 & 0 & 1 & 0 & 1 & 1 & 0 & 1\\
1 & 1 & 0 & 1 & 0 & 0 & 0 & 1\\
1 & 1 & 1 & 1 & 1 & 1 & 1 & 1\\
\hline
\end{tabular} \\
Из таблицы истинности видим: конечный результат всегда будет равен 1 - формула тавтология.
\end{document}